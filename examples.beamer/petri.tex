%% Simple Petri Net Model
%% ----------------------
%% (C) Andrew Mundy 2013
%% 
%% Displays a simple Petri net and then evolves the markings.
%% 
%% Requirements
%% ------------
%% \usepackage{tikz}
%% \usetikzlibrary{automata, chains, petri, matrices, scopes}

\begin{tikzpicture}[
	petri node/.style = {minimum size=1em},
	petri join/.style = {very thick, ->, shorten >= 1pt},
%
	place/.style = {petri node, circle, draw, very thick},
	transition/.style = {petri node, rectangle, fill},
%
	every join/.style = {petri join},
	every matrix/.style = {row sep = 1em, column sep = 2em},
	every token/.style = {fill=contentgray},	%% Remove this line if you get an error
]

	%% Draw the Petri Net
	%% Using a matrix for node placement, the & character signifies a new column, and \\ a new row.
	\matrix [] {
		\node [place] (p1) {}; & & \node [place] (p2) {};	\\
		\node [transition, label=left:$a$] (a) {}; & & \node [transition, label=right:$b$] (b) {};	\\
		\node [place] (p3) {}; & & \node [place] (p4) {};	\\
		& \node [transition, label=below:$c$] (c) {};			\\
	};

	%% Here create a chain, and use the \chainin command to add pre-existing nodes
	%% The join option draws the join between one node and the previous.
	{ [start chain=thr1]
		\chainin (p1);
		\chainin (a) [join];
		\chainin (p3) [join];
		\chainin (c) [join=by {petri join, bend right}];
		\chainin (p1) [join=by {petri join, out=90, in=0}];
	}

	{ [start chain=thr2]
		\chainin (p2);
		\chainin (b) [join];
		\chainin (p4) [join];
		\chainin (c) [join=by {petri join, bend left}];
		\chainin (p2) [join=by {petri join, out=90, in=180}];
	}

	%% Draw the markings
	%% Each marking is a list of places with a token present
	\foreach [count=\c from 1] \marking in {%
		{1,2},		   % Initial Marking
		{3,2},		   % a fired
		{3,4},		   % b fired
		{1,2},		   % c fired
		{1,4},		   % b fired
		{3,4},		   % a fired
		{1,2}%
	}{%
		\only<\c>{%
			\foreach \place in \marking {%
				\node [token] at (p\place) {};
			}
		}
	}
\end{tikzpicture}
