%% Electric Field Strength surrounding Charged Particles
%% -----------------------------------------------------
%% (C) Andrew Mundy 2013
%% 
%% Requirements
%% ------------
%% \usepackage{tikz}
%% \usepackage{etoolbox}

\begin{tikzpicture}

%% Create a list of charged particles, in the form x/y/c where x and y are co-ordinates, and c is
%% the particle charge.
\def\particles{%
	-1/0/ 1, %  1C at (-1, 0)
	 1/0/-1	 % -1C at ( 1, 0)
}

%% Calculate the electric field constant k_e
%% k_e = 1/(4\pi*\epsilon_0) -- where we will let \epsilon_0 = 1
\pgfmathsetmacro\ke{1/(4*pi)}

%% Determine the maximum possible field magnitude for hue scaling.
\def\Emax{0}
\foreach [remember=\Emax as \Emax] \x/\y/\c in \particles {
	\pgfmathsetmacro\Emax{\Emax + abs(\c)}
}
\pgfmathsetmacro\Emax{\Emax*\ke}

%% Now, iterate over the given space, at each point determining the field contribution from each 
%% particle, and using superposition to determine the resultant field magnitude and direction.  This is
%% then displayed through drawing an arrow at the  point, with the hue illustrating the field strength.
\foreach \x in {-4.0, -3.75, ..., 4.0}{%	For each x
	\foreach \y in {-2, -1.5, ..., 2.0}{%	For each y
		%% Set the resultant field strength to 0
		\pgfmathsetmacro\ERi{0}	% Resultant, i component
		\pgfmathsetmacro\ERj{0}	% Resultant, j component

		%% Ensure that this isn't the location of a particle.
		\pgfmathsetmacro\safe{1}
		\foreach [remember=\safe as \safe] \px/\py/\c in \particles {%
			\pgfmathsetmacro\safe{%
				and( \safe, not(
					and( equal( \y, \py ),
					     equal( \x, \px ) ) )
				)
			}
		}
		\pgfmathtruncatemacro\safe{\safe}

		\ifnumequal{\safe}{1}{	% Not the position of a particle, go!
			%% Now accumulate the contribution from each particle.
			\foreach [remember=\ERi as \ERi, remember=\ERj as \ERj]
				\px/\py/\pc in \particles {%
				%% Coulomb's Law
				%% -------------
				%% Calculate the vector between the current point and the charge in question
				\pgfmathsetmacro\ri{\x - \px}	% i component
				\pgfmathsetmacro\rj{\y - \py}	% j component
	
				%% Now normalise this vector
				% Magnitude
				\pgfmathsetmacro\rMAG{sqrt(\ri*\ri + \rj*\rj)}
				\pgfmathsetmacro\riN{\ri/\rMAG}	% Norm. i
				\pgfmathsetmacro\rjN{\rj/\rMAG}	% Norm. j
					
				%% We can now determine the electric field strength contribution for this particle,
				%% and then accumulate this to gain our final result.
				\pgfmathsetmacro\ei{\ke*\pc*\riN/(\rMAG*\rMAG)}
				\pgfmathsetmacro\ej{\ke*\pc*\rjN/(\rMAG*\rMAG)}

				%% And perform the accumulation
				\pgfmathsetmacro\ERi{\ERi + \ei}
				\pgfmathsetmacro\ERj{\ERj + \ej}
			}

			%% Now we have the electric field strength for this position in space.  Determine the direction of the
			%% vector, and then the required hue for the arrow.
			\pgfmathsetmacro\ERarg{atan( \ERj / \ERi )}
			\pgfmathsetmacro\ERmag{sqrt( \ERi*\ERi + \ERj*\ERj )}

			%% Determine the hue
			\pgfmathsetmacro\ERhue{1 - (\ERmag/\Emax)}
			\definecolor{elemcol}{hsb}{\ERhue,1,1}
	
			%% Draw the arrow
			\begin{scope}[xshift=\x cm, yshift=\y cm]
				\draw [thick, ->, elemcol]
					(\ERarg:-.125cm) --
					++(\ERarg:.25cm)
				;
			\end{scope}
		}{}
	}
}
\end{tikzpicture}
