%% Free Body Projection Diagram for Beamer
%% ---------------------------------------
%% (C) Andrew Mundy 2013
%%
%% Shows the displacement and velocity of a particle
%% projected with given initial velocity and displacement
%% from time t=0 until the last integral time before the
%% particle would hit the ground.
%%
%% Usage:
%% Just place in a Beamer frame.

\begin{tikzpicture}[
	xscale=0.25,
	yscale=0.20,
	ball/.style = {circle, fill=uomgrey!50!white},
	vector/.style = {thick, ->, uomgrey!50!white},
	current ball/.style = {ball, fill=blue},
	current vector/.style = {vector, blue},
	every label/.style = {font=\tiny},
]
	%% Set the initial velocity
	\def\ui{5}
	\def\uj{15}

	%% Set the initial position
	\def\si{0}
	\def\sj{0}

	%% Set the acceleration
	\def\ai{0}
	\def\aj{-9.8}

	%% Determine the point when we reach the ground, and round down for the last integral
	%% time interval to show.
	\pgfmathsetmacro\groundt{-(2*\uj)/\aj}
	\pgfmathtruncatemacro\maxt{floor(\groundt)}
	\pgfmathsetmacro\maxsi{\si + \ui*\groundt + (\ai*\groundt*\groundt)/2}

	%% Some PGF Configuration
	%% Prints floats with up to 2d.p.
	\pgfkeys{/pgf/number format/.cd,fixed,precision=2}

	%% Draw the ground
	\draw (0,0) -- (\maxsi, 0);
	\fill [pattern=north west lines, pattern color=gray] (0,0) rectangle (\maxsi, -5em);

	%% Now display the ball for times 0, 1, ..., \maxt
	\foreach [count=\c] \t in {0,...,\maxt}{%
		%% Determine the current position
		%% s = ut + 1/2at^2
		\pgfmathsetmacro\sti{\si + \ui*\t + (\ai*\t*\t)/2}
		\pgfmathsetmacro\stj{\sj + \uj*\t + (\aj*\t*\t)/2}

		%% Determine the current velocity
		\pgfmathsetmacro\vti{\ui + \ai*\t}
		\pgfmathsetmacro\vtj{\uj + \aj*\t}

		%% Convert the velocity to polar co-ordinates
		%% [1] Strictly not necessary, but cool
		\pgfmathsetmacro\argvt{atan( \vtj / \vti )}
		\pgfmathsetmacro\magvt{sqrt( \vti*\vti + \vtj*\vtj )}

		%% Save the ball's position as a co-ordinate
		\coordinate (ball-at-\t) at (\sti, \stj);

		%% Draw the ball (shadow/past)
		\pgfmathtruncatemacro\ci{\c + 1}
		\visible<\ci->{
			\node at (ball-at-\t) [ball] {};
			\draw [vector] (ball-at-\t.center) -- ++(\argvt:\magvt);
			%% Would have worked equally well (and saved us [1] above)
			% \draw [vector] (ball-at-\t.center) -- ++(\vti,\vtj);
		}

		%% Draw the ball
		\visible<\c>{ 
			\node at (ball-at-\t) [
				current ball,
				label={left:{$\vec{v}(\t) = \pgfmathprintnumber{\vti}\vec{i} +%
						\pgfmathprintnumber{\vtj}\vec{j}$}},
			] {};
			\draw [current vector] (ball-at-\t.center) -- ++(\argvt:\magvt);
			%% Would have worked equally well
			% \draw [current vector] (ball-at-\t.center) -- ++(\vti,\vtj);
		}
	}
\end{tikzpicture}
